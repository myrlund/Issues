% Først spesifiserer vi hvilken dokumentklasse vi vil ha og noen 
% globale opsjoner. Bytt ut 'article' med 'book' hvis du vil ha 
% med kapitler.
\documentclass[a4paper, twoside, titlepage, 11pt]{article}

% Så sier vi fra om hvilke tilleggspakker vi trenger
% til dokumentet vårt. De som du ikke trenger (se kommentaren) 
% kan det være en fordel å kommentere ut (sett prosenttegn foran),
% da vil kompilering gå raskere.

\usepackage[norsk]{babel}             % norske navn rundt omkring
\usepackage[T1]{fontenc}              % norsk tegnsett (æøå)
\usepackage[utf8]{inputenc}         % norsk tegnsett
\usepackage{geometry}                 % anbefalt pakke for å styre marger.

\usepackage{amsmath,amsfonts,amssymb} % matematikksymboler
\usepackage{amsthm}                   % for å lage teoremer og lignende.
\usepackage{graphics}                 % inkludering av grafikk
\usepackage{subfig}                   % hvis du vil kunne ha flere
                                      % figurer inni en figur
\usepackage{listings}                 % Fin for inkludering av kildekode

\usepackage{hyperref}                % Lager hyperlinker i evt. pdf-dokument
                                      % men har noen bugs, så den er kommentert
                                      % bort her.
                                 
% Indeksgenerering er kommentert ut her. Ta bort prosenttegnene
% hvis du vil ha en indeks:
%\usepackage{makeidx}     
%\makeindex              

% Selve dokumentet begynner:

\begin{document}

% På forsida skal vi ikke ha noen sidenummerering:

\pagestyle{empty}
\pagenumbering{roman}

% Inkluder forsida:
% Enkel forside som bruker latex sin \titlepage kommando:
% NB: Bruken av \and mellom navn!
\titlepage
\title{\large{Kravspesifikasjon} \\ \LARGE{Kvalitetsikringssystem}}
\author{Jonas Myrlund}
\date{\today}
\maketitle

% Local Variables:
% TeX-master: "master"
% End:


% Romerske tall på alt før selve rapporten starter er pent.
\pagenumbering{roman}

% For å ikke begynne innholdslista på baksida av forsida:
\cleardoublepage
% (kun aktuelt når man har twoside som global opsjon)

% Nå vi vil ha noe i topp- og bunnteksten
\pagestyle{headings}

% Si til LaTeX at vi vil ha ei innholdsliste generert akkurat her:
% \tableofcontents

% Pass på at neste side ikke begynner på baksida av en annen side.
% \cleardoublepage

% Arabisk (vanlige tall) sidenummerering. Starter på side 1 igjen.
\pagenumbering{arabic}

% Inkluder alle de andre kildefilene:

% NB: Vi trenger ikke ta med filendelsen .tex her. Den vet
%     LaTeX om selv!

\section{Overordnet funksjonalitet}
	\label{sec:overordnet}
	
	\subsection{Hovedvisning}
		\label{ssec:hovedvisning}

		For hvert prosjekt vil en liste med rutiner vises, kategorisert etter rolle, og hierarkisk vist med enkelt forelder-barn-forhold. Hver rutine skal kunne klikkes på, hvilket tar bruker til rutinevisningen (se~\ref{ssec:rutinevisning}).

	\subsection{Rutinevisning}
		\label{ssec:rutinevisning}

		blablabla

	


% \input{innledning}

% \input{bakgrunn}

% \input{resultater}

% Bibliografi/referanseliste skal komme før appendiks
\bibliography{kurs}
\bibliographystyle{plain}

% En latex-kommando for å si fra at kapitlene/seksjonene fra nå 
% av skal nummereres med store bokstaver:
\appendix

% \input{appendiks}

% Indeks for rapporten. Ta bort prosenttegn hvis du vil ha det med.
%\printindex

% Avslutter dokumentet vårt:
\end{document}

% Local Variables:
% TeX-master: "master"
% End:
